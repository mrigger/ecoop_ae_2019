\documentclass[a4paper,UKenglish]{darts-v2019}
\usepackage{microtype}
\usepackage{todonotes}
\usepackage{enumitem}
\usepackage{listings}

\newsavebox{\mybox}

\bibliographystyle{plainurl}

% Commands for artifact descriptions
% Written by Camil Demetrescu and Erik Ernst
% April 8, 2014

\newenvironment{scope}{\section{Scope}}{}
\newenvironment{content}{\section{Content}}{}
\newenvironment{getting}{\section{Getting the artifact} The artifact
endorsed by the Artifact Evaluation Committee is available free of
charge on the Dagstuhl Research Online Publication Server (DROPS).}{}
\newenvironment{platforms}{\section{Tested platforms}}{}
\newcommand{\license}[1]{{\section{License}#1}}
\newcommand{\mdsum}[1]{{\section{MD5 sum of the artifact}#1}}
\newcommand{\artifactsize}[1]{{\section{Size of the artifact}#1}}


% Author macros::begin %%%%%%%%%%%%%%%%%%%%%%%%%%%%%%%%%%%%%%%%%%%%%%%%
\title{Lifestate: Event-Driven Protocols and Callback Control Flow (Artifact)}


\author{Shawn Meier}{University of Colorado Boulder, USA\and \url{https://plv.colorado.edu/shawn/} }{shawn.meier@colorado.edu}{https://orcid.org/0000-0002-1349-4316}{}
\author{Sergio Mover}{École Polytechnique, France\and \url{http://www.sergiomover.eu/}}{sergio.mover@lix.polytechnique.fr}{https://orcid.org/0000-0003-1029-9547}{}
\author{Bor-Yuh Evan Chang}{University of Colorado Boulder, USA\and \url{https://www.cs.colorado.edu/~bec/} }{evan.chang@colorado.edu}{https://orcid.org/0000-0002-1954-0774}{}%TODO mandatory, please use full name; only 1 author per \author macro; first two parameters are mandatory, other parameters can be empty. Please provide at least the name of the affiliation and the country. The full address is optional
%\author{Joan R. Public\footnote{Optional footnote, e.g. to mark corresponding author}}{Department of Informatics, Dummy College, [optional: Address], Country}{joanrpublic@dummycollege.org}{[orcid]}{[funding]}

\authorrunning{S. Meier, S. Mover, and B.-Y.\,E. Chang}%TODO mandatory. First: Use abbreviated first/middle names. Second (only in severe cases): Use first author plus 'et al.'

\Copyright{Shawn Meier, Sergio Mover, and Bor-Yuh Evan Chang}%TODO mandatory, please use full first names. LIPIcs license is "CC-BY";  http://creativecommons.org/licenses/by/3.0/

\ccsdesc[100]{Software and its engineering~Software verification}
%\ccsdesc[100]{General and reference}%TODO mandatory: Please choose ACM 2012 classifications from https://dl.acm.org/ccs/ccs_flat.cfm

\keywords{domain-specific languages, event-based programming, language implementation, new programming models or languages, object-oriented programming, semantics, testing, verification - automated}%TODO mandatory; please add comma-separated list of keywords

%TODO Please provide information to the related scholarly information
\RelatedArticle{Shawn Meier, Sergio Mover, and Bor-Yuh Evan Chang, ``Lifestate: Event-Driven Protocols and Callback Control Flow'', in Proceedings of the European Conference On Object Oriented Programming languages (ECOOP 2019), LIPIcs, Vol.~0, pp.~0:1--0:2, 2019.\newline \url{https://doi.org/10.4230/LIPIcs.xxx.xxx.xxx}}


%Editor-only macros:: begin (do not touch as author)%%%%%%%%%%%%%%%%%%%%%%%%%%%%%%%%%%
\Volume{5}
\Issue{2}
\Article{12}
\RelatedConference{33rd European Conference on Object-Oriented Programming (ECOOP 2019), July 15--19, 2019, London, United Kingdom}
% Editor-only macros::end %%%%%%%%%%%%%%%%%%%%%%%%%%%%%%%%%%%%%%%%%%%%%%%

\makeatletter
\lst@CCPutMacro\lst@ProcessOther {"2D}{\lst@ttfamily{-{}}{-{}}}
\@empty\z@\@empty
\makeatother

\RequirePackage{relsize}
\RequirePackage{xspace}
%%% Syntax
\lstdefinelanguage{Enable}{
  morekeywords={let,in,if,then,else,new,bind,invoke,enable,disable,allow,disallow,fun,true,false,thk,event,callback,callin,force},
  literate={skip}{{$\bullet$}}1
}

\newcommand{\cbstyle}{\colordullx{0}\slshape}
\newcommand{\cistyle}{\colordullx{4}}
\newcommand{\fwktystyle}{\colordullx{1}\itshape\relsize{-1}}
\newcommand{\evtstyle}{\normalfont\ttfamily\scshape}
\lstdefinelanguage{exthighlighting}[]{highlighting}{%
  moredelim=**[is][\cbstyle]{|cb:}{|},
  moredelim=**[is][\cistyle]{|ci:}{|},
  moredelim=**[is][\fwktystyle]{|fwkty:}{|},
  moredelim=*[is][\evtstyle]{|evt:}{|}
}
\newcommand{\lifecyclepp}{\ensuremath{\text{lifecycle\raisebox{0.25ex}{\relsize{-1}+\!+}}}\xspace}
\newcommand{\codex}[2]{\ensuremath{\text{\lstinline[#1]~#2~}}}
%\newcommand{\codej}[1]{\codex{language=Java,alsolanguage=exthighlighting}{#1}}
\newcommand{\codej}[1]{\codex{language=Java}{#1}} %TODO fixme
\newcommand{\rrule}[1]{\ensuremath{\parstyle{s_{#1}}}}
\newcommand{\renable}{\ensuremath{\rightarrow}}
\newcommand{\rdisable}{\ensuremath{\nrightarrow}}
\newcommand{\rregexp}[1]{\ensuremath{\parstyle{r_{#1}}}}
\newcommand{\grregexp}[1]{\ensuremath{r_{#1}}}

\newcommand{\emptyword}{\ensuremath{\epsilon}}
\newcommand{\falseword}{\ensuremath{\emptyset}}
\newcommand{\rseq}{\ensuremath{;}}
\newcommand{\rstar}{\ensuremath{^{*}}}
\newcommand{\rand}{\ensuremath{\cap}}
\newcommand{\ror}{\ensuremath{+}}
\newcommand{\rnot}{\ensuremath{\neg}}



\begin{document}
%\input{macros}
\maketitle

\begin{abstract}
Developing interactive applications (apps) against event-driven software frameworks such as Android is notoriously difficult.
To create apps that behave as expected, developers must follow complex and often implicit \emph{asynchronous programming protocols}.
Such protocols intertwine the proper registering of callbacks to receive control from the framework with appropriate application-programming interface (API) calls that in turn affect the set of possible future callbacks.
An app violates
the protocol
when, for example, it calls a particular API method in a state of the framework where such a call is invalid.
What makes
automated reasoning hard in this domain is
largely what makes programming apps against such frameworks hard: the
specification of the protocol is unclear, and the control flow is complex, asynchronous, and higher-order.
In this paper, we tackle the problem of specifying and modeling
event-driven application-programming protocols.
In particular, we formalize a core meta-model that captures the
dialogue between event-driven frameworks and application callbacks.
Based on this
meta-model, we define a language called \emph{lifestate} that permits precise and formal descriptions of 
application-programming protocols and the callback control flow
imposed by the event-driven framework.
Lifestate unifies modeling what app callbacks can expect of the framework with specifying rules the app must respect when calling into the framework. 
In this way, we effectively combine lifecycle constraints and typestate rules.
To evaluate the effectiveness of lifestate modeling,
we provide a dynamic verification algorithm that takes as input a trace of execution of an app and a lifestate protocol specification to either produce a trace witnessing a protocol violation or a proof that no such trace is realizable.
\end{abstract}

% ARTIFACT: please stick to the structure of 7 sections provided below

% ARTIFACT: section on the scope of the artifact (what claims of the paper are intended to be backed by this artifact?)
\begin{scope}
The accompanying scholarly paper~\cite{vervita-paper} argues for a re-examination of the process of modeling callback control flow. Through this process, we identified some essential aspects of event-driven frameworks to arrive at a language called \emph{lifestates} that simultaneously captures callback control flow and event-driven application-programming protocols at the app-framework interface. This re-examination leads to both a methodology for empirically validating such event-driven framework models against corpora of app-framework interaction traces and a technique for verifying trace rearrangements are absent of protocol violations. Overall, we evaluate empirically the capacity of lifestates to model a real, complex event-driven framework like Android and the necessity of validating such models (cf. Section~6 of the accompanying paper~\cite{vervita-paper}).
This artifact includes the software and inputs that we used in the evaluation section of our paper.

\end{scope}


% ARTIFACT: section on the contents of the artifact (code, data, etc.)
\begin{content}
The artifact package includes a virtual machine image (username \texttt{verivita} and password \texttt{verivita}) with the software, trace corpora, and callback control-flow models described above, along with the measurements produced to respond to the research questions described above.
\end{content}

% ARTIFACT: section containing links to sites holding the
% latest version of the code/data, if any
\begin{getting}
% leave empty if the artifact is only available on the DROPS server.
% otherwise, provide links to the latest version of the artifact (e.g., on github)
In addition, the artifact is also available at:
\url{https://drive.google.com/open?id=15DSRQCvuxgxhYKcA7L3ah9WlQNE_t4Wr}
\end{getting}

% ARTIFACT: section specifying the platforms on which the artifact is known to
% work, including requirements beyond the operating system such as large
% amounts of memory or many processor cores
\begin{platforms}
We provide the artifact as an Ubuntu virtual machine created with Virtual Box.
We suggest using a machine with at least 16GB of RAM and dual core CPU.
\end{platforms}

% ARTIFACT: section specifying the license under which the artifact is
% made available
\license{Different parts of the artifact are under different licenses. Verivita, TraceRunner, and the trace data are under the Apache 2.0 license.  Benchtools is licensed under GPL V3.  NuXmV is under a proprietary license (see \url{https://es-static.fbk.eu/tools/nuxmv/downloads/LICENSE.txt}).}

% ARTIFACT: section specifying the md5 sum of the artifact master file
% uploaded to the Dagstuhl Research Online Publication Server, enabling
% downloaders to check that the file is the expected version and suffered
% no damage during download.
\mdsum{verivita.ova : d0ebbd97cb03e592b4d73ba57a633279}

% ARTIFACT: section specifying the size of the artifact master file uploaded
% to the Dagstuhl Research Online Publication Server
\artifactsize{14 GB}

%\subparagraph*{Acknowledgements.}
%
%I want to thank \dots

% ARTIFACT: optional appendix
\appendix

\lstset{language=bash}
\section{Running the Virtual Machine}
We suggest using VirtualBox which can be downloaded from \url{https://www.virtualbox.org/wiki/Downloads}.
The virtual machine image may be downloaded from \url{https://drive.google.com/open?id=15DSRQCvuxgxhYKcA7L3ah9WlQNE_t4Wr}.
Import the machine by clicking \url{file} $\rightarrow$ \url{import} \url{appliance} and select \url{verivita.ovf}.


\section{Extended Artifact Description}
An extended description of the artifact and how to reproduce results may be found on the desktop of the virtual machine in the file \url{artifact_description_extended.pdf}.
\bibliography{12-meier}
\end{document}
