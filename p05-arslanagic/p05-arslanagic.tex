\documentclass[a4paper,UKenglish]{darts-v2019}

 
\usepackage{microtype}

\nolinenumbers 
\usepackage{amsmath}
\usepackage{mathtools}
\usepackage{proof}
\usepackage{xspace}
\usepackage[table]{xcolor}
\usepackage[T1]{fontenc}
\usepackage{lmodern}
\usepackage{xcolor}
\usepackage{listings}


\bibliographystyle{plainurl}


\newenvironment{scope}{\section{Scope}}{}
\newenvironment{content}{\section{Content}}{}
\newenvironment{getting}{\section{Getting the artifact} The artifact 
endorsed by the Artifact Evaluation Committee is available free of 
charge on the Dagstuhl Research Online Publication Server (DROPS).}{}
\newenvironment{platforms}{\section{Tested platforms}}{}
\newcommand{\license}[1]{{\section{License}#1}}
\newcommand{\mdsum}[1]{{\section{MD5 sum of the artifact}#1}}
\newcommand{\artifactsize}[1]{{\section{Size of the artifact}#1}}


% Author macros::begin %%%%%%%%%%%%%%%%%%%%%%%%%%%%%%%%%%%%%%%%%%%%%%%% 
\newcommand{\HO}{\ensuremath{\mathsf{HO}}\xspace}
\newcommand{\secref}[1]{\S\,\ref{#1}}
\newcommand{\misty}{\textsf{MISTY}\xspace}
\title{Minimal Session Types (Artifact)} 
% Author macros::end %%%%%%%%%%%%%%%%%%%%%%%%%%%%%%%%%%%%%%%%%%%%%%%% 


\titlerunning{Minimal Session Types (Artifact)} 


\author{Alen Arslanagi\'{c}}{University of Groningen, The Netherlands}{}{https://orcid.org/0000-0002-0292-478X}{}
\author{Jorge A. P\'{e}rez}{University of Groningen, The Netherlands}{}{https://orcid.org/0000-0002-1452-6180}{}
\author{Erik Voogd}{University of Groningen, The Netherlands}{}{}{}

\authorrunning{A.\,Arslanagi\'{c}, J.\,A.\,P\'{e}rez, and E.\,Voogd}

\Copyright{Alen \,Arslanagi\'{c}, Jorge\,A.\,P\'{e}rez, and Erik\,Voogd}

\ccsdesc[100]{Theory of computation~Type structures}
\ccsdesc[500]{Theory of computation~Process calculi}
\ccsdesc[100]{Software and its engineering~Concurrent programming structures}
\ccsdesc[100]{Software and its engineering~Message passing}
%Theory of computation ? Type structures, 
%Theory of com- putation ? Process calculi, 
%Software and its engineering ? Concurrent programming structures, 
%Software and its engineering ? Message passing


\keywords{Session types, process calculi, $\pi$-calculus.}%mandatory

\RelatedArticle{}


%Editor-only macros:: begin (do not touch as author)%%%%%%%%%%%%%%%%%%%%%%%%%%%%%%%%%%
\Volume{5}
\Issue{2}
\Article{5}
\RelatedConference{33rd European Conference on Object-Oriented Programming (ECOOP 2019), July 15--19, 2019, London, United Kingdom}
% Editor-only macros::end %%%%%%%%%%%%%%%%%%%%%%%%%%%%%%%%%%%%%%%%%%%%%%%

\funding{Work partially supported by the Netherlands Organization for Scientific Research (NWO) under the VIDI Project No.\ 016.Vidi.189.046 (Unifying Correctness for Communicating Software).}

\acknowledgements{
We are grateful to the anonymous artifact reviewers for their suggestions. 
P\'{e}rez is also 
with CWI, Amsterdam and the NOVA Laboratory for Computer Science and Informatics (FCT grant NOVA LINCS PEst/UID/CEC/04516/2013), Universidade Nova de Lisboa, Portugal.}

\begin{document}

\maketitle

\begin{abstract}
This artifact contains \misty, a tool that decomposes message-passing programs with session types into programs typable with the \emph{minimal} session types we introduce in our ECOOP paper.
\misty incorporates a domain-specific language for message-passing concurrency based on a higher-order process calculus with {session types}.
Given a source program in this language, \misty follows the results in our ECOOP paper  
to produce \LaTeX\xspace code for its corresponding decomposition.  
To demonstrate the tight connection between source and decomposed programs,
\misty also allows users to simulate their corresponding reductions.
\end{abstract}

\begin{scope}
The artifact concerns \misty, a tool that demonstrates the decomposition of message-passing programs with (standard) session types into programs typable with
the \emph{minimal} session types that we define and study in our ECOOP paper. 
We have used \misty to automatically develop the several examples included in our paper. 
In our view, \misty serves as significant evidence that the conceptual benefits of relying on minimal session types, thoroughly developed in our ECOOP paper, have also concrete practical applications. 

The syntax of \misty programs closely follows Cloud Haskell
\cite{Epstein:2011:THC:2096148.2034690}. Indeed, \misty is implemented as a deeply embedded domain-specific language in 
Haskell. 
For a given \misty program, the tool generates corresponding \LaTeX\xspace code that renders the following: 
\begin{enumerate}
		\item The program's representation as an \HO process with \emph{standard} session types;
		\item The reduction chain of the \HO process obtained in (1);
		\item The decomposition of the \HO process obtained in (1) into an \HO process with \emph{minimal} session types;
		\item The reduction chain of the \HO process obtained in (3).
\end{enumerate}
\end{scope}


\begin{content}
The source code of \misty has been packaged using \texttt{stack}.
At the top level there is a \misty module that implements main \texttt{misty} and \texttt{mistymu} functions; given an input program, these functions generate the corresponding \LaTeX\xspace code. 

This module depends on the following submodules: 
\begin{itemize}
	\item \texttt{Misty.Channel} implements channel names.
	\item \texttt{Misty.Process} implements the source language for \misty as well as representations of monadic and polyadic \HO languages (target language of the decomposition).
	\item \texttt{Misty.Semantics} implements the operational semantics of the languages defined in \texttt{Process}.
	\item \texttt{Misty.Types} implements session types for input and intermediate languages as well as \emph{minimal} session types for the target language.
	\item \texttt{Misty.Decomposition} implements the decomposition function for finite processes, divided into a \emph{core fragment} and its extension with \emph{selection and branching}.	
	\item \texttt{Misty.DecompositionMu} implements the extension of the decomposition function that supports \emph{tail-recursive session types}.
	\item \texttt{Misty.DecompositionBase} contains utilities common to both decomposition functions.
\end{itemize}

The package also includes:
\begin{itemize}
\item Example \misty programs in \texttt{../examples}.
\item Already generated examples, consisting of \LaTeX\xspace code and PDF renderings.
\end{itemize}
The documentation is available in the \misty package. 
\end{content}

\begin{getting}

The latest version of our code is available on the repository: 
\begin{center}
\url{https://gitlab.com/aalen9/misty.git}
\end{center}
To set up the environment: 
\begin{itemize}
\item Install \texttt{stack} (\url{https://docs.haskellstack.org/en/stable/README/})
\item Clone the repository at \url{https://gitlab.com/aalen9/misty.git}
\end{itemize}

\end{getting}

\license{
\begin{itemize}
\item \misty is released under BSD 2-clause License (\url{https://opensource.org/licenses/ BSD-2-Clause}).
\end{itemize}
}

\begin{platforms}
The artifact has been tested 
on \texttt{macOs 10.14.3} platform, using:
\begin{itemize}
\item GHC version 8.6.4 
\item \texttt{stack} version 1.9.3 
\item \texttt{pdfLatex} \LaTeX engine for generating 
PDFs. 
\end{itemize} 
\end{platforms}

\mdsum{381ff0f71f30f9711a7afd9dd210bb04}


\artifactsize{3 MiB}

\bibliography{p05-arslanagic}


\end{document}
