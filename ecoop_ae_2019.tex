%This is a template for producing the frontmatter of an issue in DARTS.

\documentclass[a4paper,UKenglish]{dartsmaster-v2019}
 %for A4 paper format use option "a4paper", for US-letter use option "letterpaper"
 %for british hyphenation rules use option "UKenglish", for american hyphenation rules use option "USenglish"

%\graphicspath{{./graphics/}}%helpful if your graphic files are in another directory

\bibliographystyle{plainurl}% the mandatory bibstyle

\editor{Maria Christakis}{Max Planck Institute for Software Systems\\Kaiserslautern, Germany}{maria@mpi-sws.org}{}%TODO mandatory, please use full name; only 1 editor per \editor macro; first two parameters are mandatory, other parameters can be empty.

\editor{Manuel Rigger}{ETH Zurich\\Zurich, Switzerland}{manuel.rigger@inf.ethz.ch}{}

\EventEditors{Maria Christakis and Manuel Rigger}
\EventNoEds{2}
\EventLongTitle{33nd European Conference on Object-Oriented Programming (ECOOP 2019)}
\EventShortTitle{ECOOP 2019}
\EventAcronym{ECOOP}
\EventYear{2019}
\EventDate{July 15--19, 2019}
\EventLocation{London, United Kingdom}
\EventLogo{}
\SeriesVolume{5}
\SeriesIssue{2}
\ArticleNo{0} % the frontmatter is always the first paper and has always the article number 0 (zero).
\DatePublished{July 2019}


\begin{document}

\frontmatter

%%
%% PAGE 1: Cover page
%%%

\maketitle

%%
%% PAGE 2: Bibliographic data (editors, ACM classification, ISBN, license, DOI, ...)
%%

\begin{publicationinfo}%for page ii, please fill as required
\sffamily
\twocolumn

{\Large\bf\sffamily \textbf{\href{http://www.dagstuhl.de/lites}{ISSN \printISSN{}}}}

\bigskip

\newcommand{\orcid}[1]{\url{http://orcid.org/#1}}
\newcommand{\email}[1]{\href{mailto:#1}{\texttt{#1}}}

\emph{DARTS Special Issue Editors} \\[0.2cm]
\printEditorLong

\bigskip

\emph{Published online and open access by}\newline
Schloss Dagstuhl -- Leibniz-Zentrum f\"ur Informatik GmbH, Dagstuhl Publishing, Saarbr\"ucken/Wadern, Germany.

Online available at \\ \url{http://drops.dagstuhl.de/darts}.

\bigskip
\emph{Publication date}\newline
\printDatePublished{}



\bigskip

%\emph{Bibliographic information published by the Deutsche Nationalbibliothek}\newline
%The Deutsche Nationalbibliothek lists this publication in the Deutsche Nationalbibliografie; detailed bibliographic data are available in the Internet at \href{http://dnb.d-nb.de}{http://dnb.d-nb.de}.

\bigskip

\emph{License}\newline
This work is licensed under a Creative Commons Attribution 3.0 Germany license (CC BY~3.0~DE): \href{http://creativecommons.org/licenses/by/3.0/de/deed.en}{\nolinkurl{http://creativecommons.org/licenses/by/}}\linebreak \href{http://creativecommons.org/licenses/by/3.0/de/deed.en}{\nolinkurl{3.0/de/deed.en}}.
\begin{wrapfigure}[2]{l}{1.8cm}
\vspace*{-1\baselineskip}
\includegraphics[width=1.8cm]{cc-by}
\end{wrapfigure}
In brief, this license authorizes each and everybody to share (to
copy, distribute and transmit) the work under the following
conditions, without impairing or restricting the authors'
moral rights:
\begin{itemize}
\item Attribution: The work must be attributed to its authors.
\end{itemize}

The copyright is retained by the corresponding authors.

%\bigskip
\vfill
\emph{Digital Object Identifier}\newline
\printForewordDOI

\newpage

\vphantom{{\Large\bf\sffamily \textbf{\href{http://www.dagstuhl.de/lites}{ISSN \printISSN{}}}}}~~

\bigskip

\emph{Aims and Scope}\newline
The Dagstuhl Artifacts Series (DARTS) publishes evaluated research data and artifacts in all areas of computer science. An artifact can be any kind of content related to computer science research, e.g., experimental data, source code, virtual machines containing a complete setup, test suites, or tools.

%\medskip

%\bigskip

%\emph{Editorial Board}
%\begin{itemize}
%\item tba
%\end{itemize}
\vfill


\emph{Editorial Office}\newline
Michael Wagner \emph{(Managing Editor)}\\
Jutka Gasiorowski \emph{(Editorial Assistance)}\\
Dagmar Glaser \emph{(Editorial Assistance)}\\
Thomas Schillo \emph{(Technical Assistance)}

\bigskip
\emph{Contact}\newline
Schloss Dagstuhl -- Leibniz-Zentrum f\"ur Informatik\\
DARTS, Editorial Office\\
Oktavie-Allee, 66687 Wadern, Germany\\
publishing@dagstuhl.de


\bigskip

\url{http://www.dagstuhl.de/darts}

 \thispagestyle{empty}
 \onecolumn

\newpage

\end{publicationinfo}

%%
%% PAGE 5 and more: TOC etc.
%%

%\begin{dedication}%please fill or comment out
%  Insert dedication here.
%\end{dedication}


\begin{contentslist}
%To generate the table of contents copy all the .vtc files
%of the contributions to your working directory.
%For every contribution type a line
%\inputtocentry{dummycontribution}
%where the argument of \inputtocentry is the name of
%the vtc file without suffix.

%Alternatively write e.g.
\contitem
\title{Preface}
\author{John Q. Open and Joan R. Access}
\page{0:vii}

\contitem
\title{Artifact Evaluation Process}
\author{ }
\page{0:ix}

\contitem
\title{Artifact Evaluation Committee}
\author{ }
\page{0:xi}

\contitem
\title{List of Authors}
\author{ }
\page{0:xiii}

%\part{} %use if volume is divided in parts
\part{Artifacts}

\inputtocentry{darts-v2019-sample-article}

\contitem
\title{Mmmmm $\ldots$ donuts (Artifact)}
\author{Homer J. Simpson}
\page{2:1--2:23}


\end{contentslist}

\chapter{Preface} %please fill or comment out

Lorem ipsum dolor sit amet, consectetur adipiscing elit. Ut mattis
elementum fermentum. Pellentesque habitant morbi tristique senectus et
netus et malesuada fames ac turpis egestas. Nulla sapien magna,
bibendum in dictum sed, egestas vel purus. Pellentesque id ornare
lacus. Pellentesque justo elit, sodales a fringilla vitae, gravida sed
elit. Etiam turpis eros, tincidunt sit amet tempor sed, gravida quis
eros. Mauris et nunc enim. Ut congue rhoncus odio vitae lacinia. Nunc
placerat est eu eros dignissim ac tristique nisi placerat. Fusce et
hendrerit justo. Nunc feugiat pulvinar nunc ac tincidunt. Donec eu
pharetra metus. Cras malesuada ante accumsan purus dignissim
euismod. Curabitur risus ante, aliquet ut suscipit eget, vulputate non
ligula. Nullam eleifend malesuada est, nec adipiscing sapien eleifend
eget. Vestibulum fringilla diam id felis sagittis aliquet. Maecenas
sed metus vel dui vulputate pretium et id lacus. Ut eget libero augue,
ut aliquet orci. Integer sed nunc id massa interdum imperdiet. Nunc ut
consequat eros.

Morbi hendrerit dapibus augue. Proin sed adipiscing ipsum. Ut
vulputate ultricies diam id dictum. Nunc pharetra imperdiet
sodales. Morbi convallis massa vitae justo adipiscing nec congue nunc
fringilla. Pellentesque eu rhoncus ligula. Nam eros neque, hendrerit a
rhoncus vel, molestie nec nulla. Ut iaculis vulputate mauris, non
scelerisque dolor fringilla sit amet. Pellentesque habitant morbi
tristique senectus et netus et malesuada fames ac turpis
egestas. Quisque vitae accumsan risus. Sed molestie dictum
venenatis. Sed odio justo, gravida et vulputate eu, congue et lorem.

\chapter{Artifact Evaluation Process}
Editors, please describe your artifact evaluation process in this chapter:
\begin{itemize}
\item Which parties are involved?
\item How is the relation of the artifact
evaluation committee and the regular program committee of the conference?
\item How/when are decisions made and synchronized?
\end{itemize}



\begin{participants}

\chapter[Committee]{Artifact Evaluation Committee}
%use \participant for every author, eg.:

\participant Maria Christakis\\
	Max Planck Institute for Software Systems\\
	Kaiserslautern, Germany\\
	maria@mpi-sws.org

\participant Manuel Rigger\\
	ETH Zurich\\
	Zurich, Switzerland\\
	manuel.rigger@inf.ethz.ch

\participant Sara Achour\\
  MIT\\
  Cambridge, MA, USA\\
  sachour@csail.mit.edu

\participant Julia Belyakova\\
	Northeastern University\\
	Boston, MA, USA\\
	belyakova.y@northeastern.edu

\participant Junjie Chen\\
	Peking University\\
	Beijing, China\\
	chenjunjie@pku.edu.cn

\participant Marco Eilers\\
	ETH Zurich\\
	Zurich, Switzerland\\
	marco.eilers@inf.ethz.ch

\participant Juan Fumero\\
	University of Manchester\\
	Manchester, UK\\
	juan.fumero@manchester.ac.uk

\participant Tianxiao Gu\\
	Alibaba Group Holding\\
	Sunnyvale, CA, USA\\
	tianxiao.gu@gmail.com

\participant Gowtham Kaki\\
	Purdue University\\
	West Lafayette, IN, USA\\
	gowtham@purdue.edu

\participant Maria Kechagia\\
	University College London\\
	London, UK\\
	m.kechagia@ucl.ac.uk

\participant David Leopoldseder\\
	Johannes Kepler University Linz\\
	Linz, Austria\\
	david.leopoldseder@jku.at

\participant Yue Li\\
	Aarhus University\\
	Aarhus, Denmark\\
	yueli@cs.au.dk

\participant Michael Marcozzi\\
	Imperial College London\\
	London, UK\\
	m.marcozzi@imperial.ac.uk

\participant Darya Melicher\\
	Carnegie Mellon University\\
	Pittsburgh, PA, USA\\
	darya@cs.cmu.edu

\participant Lisa Nguyen Quang Do\\
	Paderborn University\\
	Paderborn, Germany\\
	lisa.nguyen@upb.de

\participant Khanh Nguyen\\
	University of California, Los Angeles\\
	Los Angeles, CA, USA\\
	khanhtn1@g.ucla.edu

\participant Burcu Kulahcioglu Ozkan\\
	Max Planck Institute for Software Systems\\
	Kaiserslautern, Germany\\
	burcu@mpi-sws.org

\participant Christian Schilling\\
	IST Austria\\
	Klosterneuburg, Austria\\
	christian.schilling@ist.ac.at

\participant Vanya Yaneva\\
	University of Edinburgh\\
	Edinburgh, UK\\
	vanya.yaneva@ed.ac.uk

% list of authors is optional
\chapter[Authors]{List of Authors}
%use \participant for every author, eg.:
\participant John Q. Public\\
  Dummy University Computing Laboratory\\
  Address, Country\\
  johnqpublic@dummyuni.org

\end{participants}


\end{document}
