%This is a template for producing the frontmatter of an issue in DARTS.

\documentclass[a4paper,UKenglish]{dartsmaster-v2019}
 %for A4 paper format use option "a4paper", for US-letter use option "letterpaper"
 %for british hyphenation rules use option "UKenglish", for american hyphenation rules use option "USenglish"

%\graphicspath{{./graphics/}}%helpful if your graphic files are in another directory

\bibliographystyle{plainurl}% the mandatory bibstyle

\editor{Maria Christakis}{Max Planck Institute for Software Systems\\Kaiserslautern, Germany}{maria@mpi-sws.org}{}%TODO mandatory, please use full name; only 1 editor per \editor macro; first two parameters are mandatory, other parameters can be empty.

\editor{Manuel Rigger}{ETH Zurich\\Zurich, Switzerland}{manuel.rigger@inf.ethz.ch}{}

\EventEditors{Maria Christakis and Manuel Rigger}
\EventNoEds{2}
\EventLongTitle{33nd European Conference on Object-Oriented Programming (ECOOP 2019)}
\EventShortTitle{ECOOP 2019}
\EventAcronym{ECOOP}
\EventYear{2019}
\EventDate{July 15--19, 2019}
\EventLocation{London, United Kingdom}
\EventLogo{}
\SeriesVolume{5}
\SeriesIssue{2}
\ArticleNo{0} % the frontmatter is always the first paper and has always the article number 0 (zero).
\DatePublished{July 2019}


\begin{document}

\frontmatter

%%
%% PAGE 1: Cover page
%%%

\maketitle

%%
%% PAGE 2: Bibliographic data (editors, ACM classification, ISBN, license, DOI, ...)
%%

\begin{publicationinfo}%for page ii, please fill as required
\sffamily
\twocolumn

{\Large\bf\sffamily \textbf{\href{http://www.dagstuhl.de/lites}{ISSN \printISSN{}}}}

\bigskip

\newcommand{\orcid}[1]{\url{http://orcid.org/#1}}
\newcommand{\email}[1]{\href{mailto:#1}{\texttt{#1}}}

\emph{DARTS Special Issue Editors} \\[0.2cm]
\printEditorLong

\bigskip

\emph{Published online and open access by}\newline
Schloss Dagstuhl -- Leibniz-Zentrum f\"ur Informatik GmbH, Dagstuhl Publishing, Saarbr\"ucken/Wadern, Germany.

Online available at \\ \url{http://drops.dagstuhl.de/darts}.

\bigskip
\emph{Publication date}\newline
\printDatePublished{}



\bigskip

%\emph{Bibliographic information published by the Deutsche Nationalbibliothek}\newline
%The Deutsche Nationalbibliothek lists this publication in the Deutsche Nationalbibliografie; detailed bibliographic data are available in the Internet at \href{http://dnb.d-nb.de}{http://dnb.d-nb.de}.

\bigskip

\emph{License}\newline
This work is licensed under a Creative Commons Attribution 3.0 Germany license (CC BY~3.0~DE): \href{http://creativecommons.org/licenses/by/3.0/de/deed.en}{\nolinkurl{http://creativecommons.org/licenses/by/}}\linebreak \href{http://creativecommons.org/licenses/by/3.0/de/deed.en}{\nolinkurl{3.0/de/deed.en}}.
\begin{wrapfigure}[2]{l}{1.8cm}
\vspace*{-1\baselineskip}
\includegraphics[width=1.8cm]{cc-by}
\end{wrapfigure}
In brief, this license authorizes each and everybody to share (to
copy, distribute and transmit) the work under the following
conditions, without impairing or restricting the authors'
moral rights:
\begin{itemize}
\item Attribution: The work must be attributed to its authors.
\end{itemize}

The copyright is retained by the corresponding authors.

%\bigskip
\vfill
\emph{Digital Object Identifier}\newline
\printForewordDOI

\newpage

\vphantom{{\Large\bf\sffamily \textbf{\href{http://www.dagstuhl.de/lites}{ISSN \printISSN{}}}}}~~

\bigskip

\emph{Aims and Scope}\newline
The Dagstuhl Artifacts Series (DARTS) publishes evaluated research data and artifacts in all areas of computer science. An artifact can be any kind of content related to computer science research, e.g., experimental data, source code, virtual machines containing a complete setup, test suites, or tools.

%\medskip

%\bigskip

%\emph{Editorial Board}
%\begin{itemize}
%\item tba
%\end{itemize}
\vfill


\emph{Editorial Office}\newline
Michael Wagner \emph{(Managing Editor)}\\
Jutka Gasiorowski \emph{(Editorial Assistance)}\\
Dagmar Glaser \emph{(Editorial Assistance)}\\
Thomas Schillo \emph{(Technical Assistance)}

\bigskip
\emph{Contact}\newline
Schloss Dagstuhl -- Leibniz-Zentrum f\"ur Informatik\\
DARTS, Editorial Office\\
Oktavie-Allee, 66687 Wadern, Germany\\
publishing@dagstuhl.de


\bigskip

\url{http://www.dagstuhl.de/darts}

 \thispagestyle{empty}
 \onecolumn

\newpage

\end{publicationinfo}

%%
%% PAGE 5 and more: TOC etc.
%%

%\begin{dedication}%please fill or comment out
%  Insert dedication here.
%\end{dedication}


\begin{contentslist}
%To generate the table of contents copy all the .vtc files
%of the contributions to your working directory.
%For every contribution type a line
%\inputtocentry{dummycontribution}
%where the argument of \inputtocentry is the name of
%the vtc file without suffix.

%Alternatively write e.g.
\contitem
\title{Preface}
\author{Maria Christakis and Manuel Rigger}
\page{0:vii}

\contitem
\title{Artifact Evaluation Process}
\author{ }
\page{0:ix}

\contitem
\title{Artifact Evaluation Committee}
\author{ }
\page{0:xi}

%\contitem
%\title{List of Authors}
%\author{ }
%\page{0:xiii}

%\part{} %use if volume is divided in parts
\part{Artifacts}

\inputtocentry{darts-v2019-sample-article}

\contitem
\title{Mmmmm $\ldots$ donuts (Artifact)}
\author{Homer J. Simpson}
\page{2:1--2:23}


\end{contentslist}

\chapter{Preface} %please fill or comment out

% What are the goals?
The \emph{Artifact Evaluation Committee} (AEC) reviews software artifacts, data sets, and proofs that accompany research papers published at ECOOP.
The goals of the \emph{Artifact Evaluation} (AE) are to ensure that the reviewed artifacts are reproducible, well-documented, and closely correspond to the associated paper.
The accepted artifacts are archived in the \emph{Dagstuhl Artifacts Series} (DARTS) published on the \emph{Dagstuhl Research Online Publication Server} (DROPS).
Each artifact is assigned a \emph{Digital Object Identifier (DOI)} that can be used in future citations.

% what is the process?
The AE process for 2019 was a continuation of the AE process of previous ECOOP editions.
In particular, the process was still based on the artifact evaluation guidelines by Shriram Krishnamurthi, Matthias Hauswirth, Steve Blackburn, and Jan Vitek published on the Artifact Evaluation site.\footnote{\url{http://www.artifact-eval.org}}
The guidelines for artifacts that contain mechanized proofs developed by the ECOOP 2018 AEC  were also reused to help both reviewers and authors in creating and reviewing such artifacts.

% what is the outcome?
This year, the committee evaluated 16 artifacts, which correspond to 57\% of all accepted papers. % total: 28 accepted papers
14 of the artifacts were accepted (a 88\% acceptance rate).
In total, 50\% of the research
papers published at ECOOP 2018 have successfully passed the AE process, indicated by an artifact-evaluation badge.
This outcome is similar to the outcomes of previous ECOOP editions; in 2018, 38\% of the research papers, and in 2017, 59\% of the research papers were accompanied by accepted artifacts.

We would like to thank the 19 members of this year's AEC, who donated their valuable time and effort to make the AE process possible.
We would also like to thank Michael Wagner for the publication of the artifacts volume, and the Program Chair Alastair Donaldson for helping us coordinate the artifact evaluation with the paper review process.
\\\\
Maria Christakis and Manuel Rigger\\
(Artifact Evaluation Co-Chairs)

\chapter{Artifact Evaluation Process}

The authors of all papers that were accepted to ECOOP 2019 had the option to submit an artifact with their paper, irrespective of the paper category (i.e., Research, Tool Insights, Reproduction Study, Experience Report, Pearl, and Brave New Idea).
Each artifact was evaluated by three reviewers who were part of the AEC.
The reviewing process consisted of two phases.
In the ``kick-the-tires'' phase, reviewers briefly verified the basic integrity, documentation, and set-up of the artifacts.
In case of any issues, reviewers had the opportunity to ask clarifying questions to the authors.
Authors, in turn, could respond to the reviewers' first feedback, and provide missing documentation or small fixes to the artifacts, to ensure that reviewers were able to fully evaluate the artifacts.
In the second phase, each reviewer had three weeks to do a comprehensive evaluation of the two to three artifacts they were assigned to review.
This included assessing whether an artifact fully corresponded to the paper, whether all results presented in the paper could be reproduced, how well the artifact was documented, and how easy it would be to re-use the artifact in future research.
The review phase was then followed by a discussion phase, in which artifacts were discussed to converge on either the artifacts' acceptance or rejection.
Authors that received an acceptance notification were given one week of time to incorporate reviewers' feedback and submit the camera-ready version of their artifacts.

\begin{participants}

\chapter[Committee]{Artifact Evaluation Committee}
%use \participant for every author, eg.:

\participant Maria Christakis\\
	Max Planck Institute for Software Systems\\
	Saarbrücken and Kaiserslautern, Germany\\
	maria@mpi-sws.org

\participant Manuel Rigger\\
	ETH Zurich\\
	Zurich, Switzerland\\
	manuel.rigger@inf.ethz.ch

\participant Sara Achour\\
  MIT\\
  Cambridge, MA, USA\\
  sachour@csail.mit.edu

\participant Julia Belyakova\\
	Northeastern University\\
	Boston, MA, USA\\
	belyakova.y@northeastern.edu

\participant Junjie Chen\\
	Peking University\\
	Beijing, China\\
	chenjunjie@pku.edu.cn

\participant Marco Eilers\\
	ETH Zurich\\
	Zurich, Switzerland\\
	marco.eilers@inf.ethz.ch

\participant Juan Fumero\\
	University of Manchester\\
	Manchester, UK\\
	juan.fumero@manchester.ac.uk

\participant Tianxiao Gu\\
	Alibaba Group\\
	Sunnyvale, CA, USA\\
	tianxiao.gu@alibaba-inc.com

\participant Gowtham Kaki\\
	Purdue University\\
	West Lafayette, IN, USA\\
	gowtham@purdue.edu

\participant Maria Kechagia\\
	University College London\\
	London, UK\\
	m.kechagia@ucl.ac.uk

\participant David Leopoldseder\\
	Johannes Kepler University Linz\\
	Linz, Austria\\
	david.leopoldseder@jku.at

\participant Yue Li\\
	Aarhus University\\
	Aarhus, Denmark\\
	yueli@cs.au.dk

\participant Michael Marcozzi\\
	Imperial College London\\
	London, UK\\
	m.marcozzi@imperial.ac.uk

\participant Darya Melicher\\
	Carnegie Mellon University\\
	Pittsburgh, PA, USA\\
	darya@cs.cmu.edu

\participant Lisa Nguyen Quang Do\\
	Paderborn University\\
	Paderborn, Germany\\
	lisa.nguyen@upb.de

\participant Khanh Nguyen\\
	University of California, Los Angeles\\
	Los Angeles, CA, USA\\
	khanhtn1@g.ucla.edu

\participant Burcu Kulahcioglu Ozkan\\
	Max Planck Institute for Software Systems\\
	Saarbrücken and Kaiserslautern, Germany\\
	burcu@mpi-sws.org

\participant Christian Schilling\\
	IST Austria\\
	Klosterneuburg, Austria\\
	christian.schilling@ist.ac.at

\participant Vanya Yaneva\\
	University of Edinburgh\\
	Edinburgh, UK\\
	vanya.yaneva@ed.ac.uk

\end{participants}


\end{document}
