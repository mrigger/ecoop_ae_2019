\documentclass{darts-v2019}
\title{On Julia’s Efficient Algorithm for Subtyping Union Types and Covariant Tuples (Artifact)}

\Volume{5}
\Issue{2}
\Article{8}
\RelatedConference{33rd European Conference on Object-Oriented Programming (ECOOP 2019), July 15--19, 2019, London, United Kingdom}

\newenvironment{scope}{\section{Scope}}{}
\newenvironment{content}{\section{Content}}{}
\newenvironment{getting}{\section{Getting the artifact} The artifact 
endorsed by the Artifact Evaluation Committee is available free of 
charge on the Dagstuhl Research Online Publication Server (DROPS).}{}
\newenvironment{platforms}{\section{Tested platforms}}{}
\newcommand{\license}[1]{{\section{License}#1}}
\newcommand{\mdsum}[1]{{\section{MD5 sum of the artifact}#1}}
\newcommand{\artifactsize}[1]{{\section{Size of the artifact}#1}}

\author{Benjamin Chung}{Northeastern University}{bchung@ccs.neu.edu}{}{}
\author{Francesco Zappa Nardelli}{Inria}{francesco.zappa\_nardelli@inria.fr}{}{}
\author{Jan Vitek}{Northeastern University \and Czech Technical University in Prague}{j.vitek@neu.edu}{}{}
\authorrunning{B.\,Chung\, F.\,Zappa Nardelli and J.\,Vitek}

\Copyright{Benjamin W. Chung, Francesco Zappa Nardelli, and Jan Vitek}

\ccsdesc[500]{Theory of computation~Type theory}
\ccsdesc[300]{Software and its engineering~General programming languages}

\keywords{Type systems, subtyping, algorithmic type systems, distributive unions}

\begin{document}
\maketitle

\section{Introduction}

This is the artifact for the pearl paper "On Julia’s efficient algorithm for subtyping union types and covariant tuples." It consists of two primary components:

\begin{itemize}
\item \texttt{index.html}: An implementation of the subtyping algorithm running in a webpage. This implementation is modified only slightly from the one described in the paper to enable visualization. For sources, see the \texttt{web-impl} directory.
\item \texttt{julia-iterators.v}: The Coq source code for the proofs referenced in our paper.
\end{itemize}

This document is a worse-formatted and non-executable version of
\texttt{index.html}. We suggest the the online version (at
https://benchung.github.io/subtype-artifact/) or simply open
\texttt{index.html} from the artifact archive for information on the artifact
and to try out our algorithm. The website version of the artifact is tested to
work in Google Chrome, and should work in all modern browsers.

\section{Proof}

The proof script (found in \texttt{proof/julia-iterators.v}) depends on Coq 8.9.0. A detailed description of our 
proof can be found in section 3 of the paper. 
The proof is standalone, and has no library dependencies. 

It relies on the standard library provided axiom \verb|Eqdep.Eq_rect_eq.eq_rect_eq|, which establishes the invariance
under substitution of dependent equality. In our formalization, structural type
iterators are dependent upon the type over which they iterate. We rely on this
axiom to decide when two iterators are iterating over the same or different types.
It is an axiom in our system as it is independent of the calculus of constructions.

\section{Implementation}

We include a web implementation of our algorithm. To use it, please see \texttt{index.html} for
the running implementation and instructions on its use and compilation.

\subsection{Compiling the Implementation}

The implementation is written in OCaml and compiled using js\_of\_ocaml. It requires:
\begin{itemize}
\item OCaml 4.07.0 or later
\item opam 2.0.4 or later
\end{itemize}

To compile the OCaml to Javascript, run

\begin{verbatim}
make deps
make
\end{verbatim}

in the \texttt{web-impl} subdirectory, which should update the file \texttt{web-impl/js/subtype.js}.

\section{License}

Copyright 2019 Benjamin Chung, Francesco Zappa Nardelli, Jan Vitek

The artifact associated with this description is licensed under the Apache
License, Version 2.0 (the "License"); you may not use the associated artifact
except in compliance with the License. You may obtain a copy of the License at

    http://www.apache.org/licenses/LICENSE-2.0

Unless required by applicable law or agreed to in writing, software
distributed under the License is distributed on an "AS IS" BASIS,
WITHOUT WARRANTIES OR CONDITIONS OF ANY KIND, either express or implied.
See the License for the specific language governing permissions and
limitations under the License.

\end{document}
