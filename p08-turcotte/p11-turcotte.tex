\documentclass[a4paper,UKenglish]{darts-v2019}
 
\usepackage{microtype}%if unwanted, comment out or use option "draft"

%\graphicspath{{./graphics/}}%helpful if your graphic files are in another directory

%\nolinenumbers to disable line numbers

\bibliographystyle{plainurl}% the mandatory bibstyle

\Copyright{Alexi Turcotte, Ellen Arteca, and Gregor Richards}


\newenvironment{scope}{\section{Scope}}{}
\newenvironment{content}{\section{Content}}{}
\newenvironment{getting}{\section{Getting the artifact} The artifact 
endorsed by the Artifact Evaluation Committee is available free of 
charge on the Dagstuhl Research Online Publication Server (DROPS).}{}
\newenvironment{platforms}{\section{Tested platforms}}{}
\newcommand{\license}[1]{{\section{License}#1}}
\newcommand{\mdsum}[1]{{\section{MD5 sum of the artifact}#1}}
\newcommand{\artifactsize}[1]{{\section{Size of the artifact}#1}}


% Author macros::begin %%%%%%%%%%%%%%%%%%%%%%%%%%%%%%%%%%%%%%%%%%%%%%%%
\title{Reasoning About Foreign Function Interfaces Without Modelling the Foreign Language (Artifact)}

\titlerunning{Reasoning About Foreign Function Interfaces (Artifact)} 

\author{Alexi Turcotte}{Northeastern University, USA}{}{}{}

\author{Ellen Arteca}{Northeastern University, USA}{}{}{}

\author{Gregor Richards}{University of Waterloo, Canada}{}{}{}

\authorrunning{A. Turcotte, E. Arteca, and G. Richards}

\ccsdesc[500]{Software and its engineering~Interoperability}
\ccsdesc[500]{Software and its engineering~Semantics}

\keywords{Language design, Language interoperation, Formal semantics}

%TODO Please provide information to the related scholarly information
\RelatedArticle{To be filled in}

%Editor-only macros:: begin (do not touch as author)%%%%%%%%%%%%%%%%%%%%%%%%%%%%%%%%%%
\Volume{3}
\Issue{2}
\Article{1}
\RelatedConference{33rd European Conference on Object-Oriented Programming (ECOOP '19), July 17--18, 2019, Hammersmith, London, United Kingdom}
% Editor-only macros::end %%%%%%%%%%%%%%%%%%%%%%%%%%%%%%%%%%%%%%%%%%%%%%%

\begin{document}

\maketitle

\begin{abstract}
There are two components to this artifact.
First, a we provide a mechanization of the formalization in the paper, as well as mechanized proofs of the main results from the paper.
Second, we provide a full implementation of Poseidon Lua, the language implemented in the paper.
Instructions for all components of the artifact are included this document.

\end{abstract}

\begin{scope}

This artifact is intended to mimic our semantics and to substantiate our assertions about the semantics, as well as provide a concrete implementation of Poseidon Lua.

First, we provide a Coq mechanization which matches the formal semantics presented in the paper, and we mechanized the theorems from the paper and proved them in Coq.
The correspondence of the mechanized semantics to the semantics presented in the paper can be found in the artifact itself.

The second aspect of this artifact is an implementation of Poseidon Lua, the language presented in the paper.
We include the language itself in the artifact, so as to allow users to write and test some programs.

\end{scope}

% ARTIFACT: section on the contents of the artifact (code, data, etc.)
\begin{content}
There are two main components to this artifact.
First, we provide a Coq mechanization of the semantics and theorems presented in the paper.
The proof is located in the directory \textbf{/home/reviewer/Desktop/clua}.
Proof compilation instructions are available in the artifact itself.

Second, we provide an implementation of Poseidon Lua.
The source code is included in the directory: \textbf{/home/reviewer/Desktop/Poseidon\_Lua}
The setup to run the language is all done already on this VM; bash is already configured with the necessary environment variables.
There are numerous pre-written examples in the artifact. 

There is also a README in the Posidon\_Lua directory, which provides a more detailed explanation.

\end{content}

\begin{getting}
The VM is a Debian VirtualBox.
Click
\href{https://drive.google.com/file/d/1NkZr-Dx7RmCuISKGdry1fyTG0lXYovxR/view?usp=sharing}{\textbf{\color{blue}{here}}}
to download the VM.

The login information is as follows:
\begin{itemize}
    \item Username: {\bf reviewer}
    \item User password: {\bf review} (and user {\bf reviewer} has sudo)
    \item Root password: {\bf password}
\end{itemize}
\end{getting}

\begin{platforms}
This artifact is a VirtualBox virtual machine, so it can run on any platform.
As with any virtual machine, it will run faster if there is more RAM available, but this is not necessary.
Note that the virtual machine is running Debian, so basic familiarity with Linux would help when using the artifact.

The proof compiles on Coq version 8.9.1, which is the version installed on the VM.
\end{platforms}

\license{The Coq mechanization is licensed under the MIT license.

Poseidon Lua is modifications to Lua and Typed Lua and is under their respective licenses. Both are licensed under the MIT license.

It's also important to note that the rest of the software on the system is under various licenses, as it's a complete OS image, which can be found in /usr/share/doc/*/copyright.}

\mdsum{007f2651fe2c6919e8423123d4bc2db5}

\artifactsize{2.61 GiB (2.8 GB)}

\subparagraph*{Acknowledgements.}

The authors would like to thank Rafi Turas for writing the implementation of these techniques in Poseidon Lua. 
We’d also like to thank Hugo Musso Gualandi for his valuable discussions/feedback. This work was partially funded by NSERC.


\end{document}
