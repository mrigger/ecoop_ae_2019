\documentclass[a4paper,english]{darts-v2019}
%This is a template for producing DARTS artifact descriptions.
%for A4 paper format use option "a4paper", for US-letter use option "letterpaper"
%for british hyphenation rules use option "UKenglish", for american hyphenation rules use option "USenglish"
% for section-numbered lemmas etc., use "numberwithinsect"

\usepackage{microtype}%if unwanted, comment out or use option "draft"

%\graphicspath{{./graphics/}}%helpful if your graphic files are in another directory

%\nolinenumbers to disable line numbers

\bibliographystyle{plainurl}% the mandatory bibstyle

% Commands for artifact descriptions
% Written by Camil Demetrescu and Erik Ernst
% April 8, 2014

\newenvironment{scope}{\section{Scope}}{}
\newenvironment{content}{\section{Content}}{}
\newenvironment{getting}{\section{Getting the artifact} The artifact
endorsed by the Artifact Evaluation Committee is available free of
charge on the Dagstuhl Research Online Publication Server (DROPS).}{}
\newenvironment{platforms}{\section{Tested platforms}}{}
\newcommand{\license}[1]{{\section{License}#1}}
\newcommand{\mdsum}[1]{{\section{MD5 sum of the artifact}#1}}
\newcommand{\artifactsize}[1]{{\section{Size of the artifact}#1}}


% Author macros::begin %%%%%%%%%%%%%%%%%%%%%%%%%%%%%%%%%%%%%%%%%%%%%%%%
\title{Automated Large-scale Multi-language Dynamic Program Analysis in the Wild (Artifact)} %TODO Please add

%\titlerunning{A Sample DARTS Research Description} %optional, in case that the title is too long; the running title should fit into the top page column

% ARTIFACT: Authors may not be exactly the same as the related scholarly paper, e.g., you may want to include authors who contributed to the preparation of the artifact, but not to the companion paper

\author{Alex Villaz\'on}{Universidad Privada Boliviana, Bolivia}{avillazon@upb.edu}{https://orcid.org/0000-0001-8428-3420}{}%TODO mandatory, please use full name; only 1 author per \author macro; first two parameters are mandatory, other parameters can be empty. Please provide at least the name of the affiliation and the country. The full address is optional

\author{Haiyang Sun}{Universit\`a della Svizzera italiana, Switzerland}{haiyang.sun@usi.ch}{}{}
\author{Andrea Ros\`a}{Universit\`a della Svizzera italiana, Switzerland}{andrea.rosa@usi.ch}{}{}
\author{Eduardo Rosales}{Universit\`a della Svizzera italiana, Switzerland}{rosale@usi.ch}{https://orcid.org/0000-0002-6404-3128}{}
\author{Daniele Bonetta}{Oracle Labs, United States}{daniele.bonetta@oracle.com}{}{}
\author{Isabella Defilippis}{Universidad Privada Boliviana, Bolivia}{isabelladefilippis@upb.edu}{}{}
\author{Sergio Oporto}{Universidad Privada Boliviana, Bolivia}{sergiooporto@upb.edu}{}{}
\author{Walter Binder}{Universit\`a della Svizzera italiana, Switzerland}{walter.binder@usi.ch}{}{}



\authorrunning{A.~Villaz\'on et al.}%TODO mandatory. First: Use abbreviated first/middle names. Second (only in severe cases): Use first author plus 'et al.'

\Copyright{Alex Villaz\'on, Haiyang Sun, Andrea Ros\`a, Eduardo Rosales, Daniele Bonetta, Isabella Defilippis, Sergio Oporto, and Walter Binder}%TODO mandatory, please use full first names. LIPIcs license is "CC-BY";  http://creativecommons.org/licenses/by/3.0/

\ccsdesc[500]{Software and its engineering~Dynamic analysis}
%\ccsdesc[100]{General and reference}%TODO mandatory: Please choose ACM 2012 classifications from https://dl.acm.org/ccs/ccs_flat.cfm

\keywords{Dynamic program analysis, code repositories, GitHub, Node.js, Java, Scala, promises, JIT-unfriendly code, task granularity}%TODO mandatory; please add comma-separated list of keywords

%TODO Please provide information to the related scholarly information
\RelatedArticle{}

%Editor-only macros:: begin (do not touch as author)%%%%%%%%%%%%%%%%%%%%%%%%%%%%%%%%%%
\Volume{5}
\Issue{2}
\Article{11}
\RelatedConference{33rd European Conference on Object-Oriented Programming (ECOOP 2019), July 15--19, 2019, London, United Kingdom}
% Editor-only macros::end %%%%%%%%%%%%%%%%%%%%%%%%%%%%%%%%%%%%%%%%%%%%%%%

\begin{document}

\maketitle

\begin{abstract}
This artifact provides a preliminary release of NAB, a distributed infrastructure for executing large-scale dynamic program analyses (DPAs). The artifact consists of ready-to-use Docker containers that allow one to run different DPA tools (Deep-Promise, JITProf, and tgp) on Node.js, Java, and Scala projects hosted on GitHub. The artifact enables the reproduction of the figures and tables of the related paper ``Automated Large-scale Multi-language Dynamic Program Analysis in the Wild'' with pre-collected data (several GBs) and the execution of DPAs on specific sets of GitHub projects.
 \end{abstract}

% ARTIFACT: please stick to the structure of 7 sections provided below

% ARTIFACT: section on the scope of the artifact (what claims of the paper are intended to be backed by this artifact?)
\begin{scope}

The artifact contains a preliminary release of NAB, including different services (i.e., NAB-Master, NAB-Crawler, NAB-Analyzer, and NAB-Dashboard) that run in Docker containers, as described in the related paper  ``Automated Large-scale Multi-language Dynamic Program Analysis in the Wild''~\cite{Villazon:2019:NAB}. The NAB-Analyzer Docker image in the artifact includes three DPA tools (Deep-Promise~\cite{Villazon:2019:NAB}, JITProf \cite{Gong:2015:JITProf} and tgp~\cite{tgp2017,Rosa:2018:tgp,TOPLAS19}
and two runtimes (Oracle's JVM~\cite{HotSpotJVM} and GraalVM~\cite{graaljs2017}) for analyzing open-source Java, Scala, and Node.js projects hosted on GitHub.

To reproduce the figures and tables from the related paper, we use pre-collected data (i.e., results of DPA tools applied to more than 56K  Node.js, Java, and Scala projects from GitHub) stored in MongoDB databases. The pre-collected data is stored in the NAB-Dashboard image. We provide scripts to restore the database, to process the results, and to generate the tables and figures.

The artifact also includes scripts to execute the DPA tools on the full set of GitHub projects used to generate the pre-collected data (i.e., to reproduce the experiments), as well as to execute the DPA tools on smaller subsets of selected GitHub projects (for evaluating NAB on a smaller input set).  The scripts automate different activities, such as cloning, building, running test code applying the selected DPA tool, collecting DPA results, and generating output in the form of tables and figures.

\end{scope}

% ARTIFACT: section on the contents of the artifact (code, data, etc.)
\begin{content}
The artifact includes the following content:
\begin{itemize}
\item {\tt docker/}: directory containing four Docker images, one for each NAB core service ({\tt nab-master.img}, {\tt nab-analyzer.img}, {\tt nab-crawler.img}, and {\tt nab-dashboard.img}).
\item {\tt nab/}: directory containing scripts, configuration files for each case study, sample configuration files, and NAB service deployment files.
\item {\tt utils/}: directory containing helper scripts.
\item {\tt load.sh}: script to load the images of NAB services in the Docker installation.
\item {\tt README.pdf}: a description about how to run the Docker images containing NAB services for the three DPAs covered in the related paper.
\item {\tt run-nab.sh}: the main script to start the core NAB services and run the experiments.


\end{itemize}
\end{content}

% ARTIFACT: section containing links to sites holding the
% latest version of the code/data, if any
\begin{getting}
% leave empty if the artifact is only available on the DROPS server.
% otherwise, provide links to the latest version of the artifact (e.g., on github)
In addition, the artifact is also available at:
\url{http://research.upb.edu/NAB/nab-artifact.tgz}.
\end{getting}

% ARTIFACT: section specifying the platforms on which the artifact is known to
% work, including requirements beyond the operating system such as large
% amounts of memory or many processor cores
\begin{platforms}
The artifact has been successfully tested on machines running Linux or macOS with at least 16~GB of RAM and 50~GB of free storage. It was tested with Docker~18.06.1-ce (build e68fc7a) and 18.09.2 (build 6247962).
\end{platforms}

% ARTIFACT: section specifying the license under which the artifact is
% made available
%\license{The artifact is available under license EPL-1.0 (\url{https://www.eclipse.org/legal/epl-v10.html}).}
\license{The artifact is available under Apache License 2.0.}

% ARTIFACT: section specifying the md5 sum of the artifact master file
% uploaded to the Dagstuhl Research Online Publication Server, enabling
% downloaders to check that the file is the expected version and suffered
% no damage during download.
\mdsum{fa2151a3fa429d676e7eb2c77d5aad62}

% ARTIFACT: section specifying the size of the artifact master file uploaded
% to the Dagstuhl Research Online Publication Server
\artifactsize{5.6 GB}

\subparagraph*{Acknowledgements.}
This work has been supported by Oracle (ERO project 1332), Swiss National Science Foundation (scientific exchange project IZSEZ0\_177215), Hasler Foundation (project~18012), and by a Bridging Grant with Japan (BG~04-122017).

% ARTIFACT: include here any additional references, if needed...

%%
%% Bibliography
%%

%% Either use bibtex (recommended),

\bibliography{p11-villazon}

%% .. or use the thebibliography environment explicitely

\end{document}
